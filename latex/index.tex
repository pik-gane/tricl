\section*{Introduction }

Tri\+Cl is a tool for running simulations of various kinds of social dynamics, modeled by specifying when certain types of relationships between two entities of certain types form or vanish or certain types actions between two entities happen.

This is the code documentation automatically generated by doxygen and is meant as a resource for code developers (Model developers please see \href{https://github.com/mensch72/tricl}{\tt the R\+E\+A\+D\+M\+E.\+md file in the tricl github repository}).

Tri\+Cl is implemented in C++. Its top-\/level code file is \hyperlink{tricl_8cpp}{tricl.\+cpp}, its main function is \hyperlink{tricl_8cpp_a0ddf1224851353fc92bfbff6f499fa97}{main()}.

\section*{T\+O\+DO }

model capabilities\+:
\begin{DoxyItemize}
\item allow legs to influence termination
\item allow legs to influence success prob. of establishment
\item allow legs to attempt establishment
\item add actions
\end{DoxyItemize}

simulation options\+:
\begin{DoxyItemize}
\item as an alternative to tmax and max\+\_\+n\+\_\+events, add max\+\_\+wall
\item add a particle filtering mode
\end{DoxyItemize}

network theory stuff\+:
\begin{DoxyItemize}
\item output triangle counts clustering coefficient(s)
\item add more random network models
\end{DoxyItemize}

convenience\+:
\begin{DoxyItemize}
\item support custom prefixes for entity label generation
\item check whether in \hyperlink{finish_8cpp_a6dfe1abe0d1eb3ddc1ca081de98b5342}{finish()}, current\+\_\+t = max\+\_\+t is correct
\item metaparameters\+: identifier\+: \mbox{[}value, description\mbox{]} --$>$ cmd. line help
\item add auxiliaries\+: identifier\+: expression
\end{DoxyItemize}

optimization\+:
\begin{DoxyItemize}
\item replace (ordered) map t2ev by unordered\+\_\+map plus additional variable next\+\_\+t as described in \href{https://www.geeksforgeeks.org/design-a-stack-to-retrieve-original-elements-and-return-the-minimum-element-in-o1-time-and-o1-space/?ref=rp}{\tt https\+://www.\+geeksforgeeks.\+org/design-\/a-\/stack-\/to-\/retrieve-\/original-\/elements-\/and-\/return-\/the-\/minimum-\/element-\/in-\/o1-\/time-\/and-\/o1-\/space/?ref=rp}
\item use const args as much as possible in inner loops (?)
\item inline most called functions
\item speed up probability2probunits by precomputing probability for events with constant success probability
\item think of partial parallelization
\end{DoxyItemize}

input/output\+:
\begin{DoxyItemize}
\item include metadata into gexf files
\item redirect messages to log file
\item support \char`\"{}premerged\char`\"{} edges in gexf via $<$spells$>$$<$spell start=\char`\"{}...\char`\"{} end=\char`\"{}...\char`\"{}$>$$<$/spell$>$...$<$/spells$>$, by a postprocessing step that firts sorts and then merges the edges in the file by doing {\ttfamily gunzip -\/c myoutput.\+gexf.\+gz $\vert$ sort -\/t\char`\"{}-\/\char`\"{} -\/k3 $\vert$ mergelinks $\vert$ gzip $>$ mysortedoutput.\+gexf.\+gz} for which we need to provide a binary \char`\"{}mergelinks\char`\"{}
\item support prespecified positions in visualization, see here\+: \href{https://github.com/gephi/gephi/issues/2038}{\tt https\+://github.\+com/gephi/gephi/issues/2038}
\item support entity type detection from columns in csv file
\end{DoxyItemize}

model estimation\+:
\begin{DoxyItemize}
\item add command line options --events=events.\+csv, --logl and --grad
\item if given, read lines from events.\+csv rather than pop\+\_\+next\+\_\+event
\item if --logl, accumulate loglikelhood instead of scheduling and write only logl to stdout
\item if --grad, also accumulate and output gradient w.\+r.\+t. metaparameters
\item write python template script for estimating metaparameters by max.\+likelihood method, using scipy.\+optimize
\end{DoxyItemize}

\section*{Terminology }


\begin{DoxyItemize}
\item {\bfseries entity} type\+: a type of concrete or abstract entity (see below), e.\+g. \char`\"{}an individual\char`\"{}, \char`\"{}a news channel\char`\"{}, \char`\"{}a social group\char`\"{}, \char`\"{}an opinion\char`\"{}, or \char`\"{}an infection state\char`\"{}
\item {\bfseries entity\+:} any concrete or abstract object that can stand in a relationship with other entities, e.\+g. \char`\"{}\+John\char`\"{}, \char`\"{}the B\+B\+C\char`\"{}, \char`\"{}catholics\char`\"{}, \char`\"{}\+Elvis lives\char`\"{}, or \char`\"{}infected with Dengue\char`\"{}
\item {\bfseries relationship} {\bfseries type\+:} any concrete or abstract type of directed relationship two entities can stand in, e.\+g. \char`\"{}is friends with\char`\"{}, \char`\"{}is a subscriber of\char`\"{}, \char`\"{}belongs to\char`\"{}, \char`\"{}holds\char`\"{}, or \char`\"{}is\char`\"{}. the special relationship type \char`\"{}=\char`\"{} (with id R\+T\+\_\+\+ID) encodes the identity of an entity with itself.
\item {\bfseries action} {\bfseries type\+:} any type of thing that can happen at a singular time point between two entities, e.\+g. \char`\"{}kisses\char`\"{} or \char`\"{}utters that\char`\"{} (not implemented yet)
\item {\bfseries act\+:} a pair of entities plus an action type plus a time-\/point. we say the act \char`\"{}occurs\char`\"{} at that time.
\item {\bfseries link} {\bfseries type\+:} a pair of entity types plus a relationship or action type, e.\+g. \char`\"{}an individual -\/ is a subscriber of -\/ a news channel\char`\"{} or \char`\"{}an individual -\/ utters -\/ an opinion\char`\"{}
\item {\bfseries link\+:} a pair of entities plus a relationship or action type, e.\+g. \char`\"{}\+John -\/ is a subscriber of -\/ the B\+B\+C\char`\"{}. if it has a relationship type, a link \char`\"{}exists\char`\"{} as long as the entities stand in the respective relationship. if it has an action type, a link \char`\"{}flashes\char`\"{} whenever a corresponding act occurs. the source entity id is called \char`\"{}e1\char`\"{} in code, the destination entity id \char`\"{}e3\char`\"{} (because of angles, see below)
\item {\bfseries relationship\+:} a link of relationship type
\item {\bfseries action\+:} a link of action type (not to be confused with an act!)
\item {\bfseries impact} of an action\+: a nonnegative real number that is increased by one whenever the link flashes and decays exponentially at a certain rate. a link\textquotesingle{}s impact determines the link\textquotesingle{}s influence on adjacent events.
\item {\bfseries event\+:} an event class plus a link, e.\+g.
\begin{DoxyItemize}
\item \char`\"{}establishment of the link \textquotesingle{}\+John -\/ is a subscriber of -\/ the B\+B\+C\textquotesingle{}\char`\"{} (event class E\+C\+\_\+\+E\+ST)
\item \char`\"{}termination of the link \textquotesingle{}\+John -\/ is friends with -\/ Alice\textquotesingle{}\char`\"{} (event class E\+C\+\_\+\+T\+E\+RM)
\item \char`\"{}occurrence of the act \textquotesingle{}\+Alice -\/ utters that -\/ Elvis lives\textquotesingle{}\char`\"{} (event class E\+C\+\_\+\+A\+CT)
\end{DoxyItemize}
\item {\bfseries leg\+:} a relationship or action type rat plus an entity. a leg can influence an adjacent termination event but no establishment or act occurrence events
\item {\bfseries angle\+:} a middle entity plus pair of relationship or action types. an angle can influence any adjacent event. an angle can also encode a leg if either relationship or action type equals the special value N\+O\+\_\+\+RT
\item {\bfseries influence\+:} a event plus a leg or angle that influences it (or the special value N\+O\+\_\+\+A\+N\+G\+LE if the event happens spontaneously)
\item {\bfseries attempt} {\bfseries rate} of an influence\+: the probability rate with which the influence attempts the corresponding event
\item {\bfseries success} {\bfseries probability} {\bfseries units} of an event\+: probability that an attempted event actually happens, transformed via a sigmoidal function.
\end{DoxyItemize}

The following diagrams show the relationships between these types of things\+: \begin{DoxyVerb}  e1 –––––––––rat13–––––––––> e3
entity     relationship     entity
  .       or action type       .
  .             .              .
  .       \______________________/
  .           an out-leg for e1
  .             .              .
\______________________/       .
    an in-leg for e3           .
  .             .              .
\________________________________/
              a link


  e1 ––rat12––> e2 ––rat23––> e3
 \______________________________/
             an angle
\end{DoxyVerb}


\section*{Naming conventions }

\subsection*{Abbreviations used in variable naming }


\begin{DoxyItemize}
\item {\bfseries ar\+:} attempt rate
\item {\bfseries spu\+:} success probability units
\item {\bfseries e\+:} entity
\item {\bfseries e1\+:} source entity of a link or event
\item {\bfseries e2\+:} middle entity of an angle or influence, or \char`\"{}other\char`\"{} entity of a leg
\item {\bfseries e3\+:} target entity of a link or event
\item {\bfseries et\+:} entity type
\item {\bfseries et1\+:} source entity type of a link or event type
\item {\bfseries et2\+:} middle entity type of an angle or influence type, or other entity type of a leg type
\item {\bfseries et3\+:} target entity type of a link or event type
\item {\bfseries ev\+:} event
\item {\bfseries evt\+:} event type
\item {\bfseries infl\+:} influence
\item {\bfseries inflt\+:} influence type
\item {\bfseries l\+:} link
\item {\bfseries lt\+:} link type
\item {\bfseries rat\+:} relationship or action type
\item {\bfseries rat13\+:} type of rel. or action from source to target of a link or event
\item {\bfseries rat12\+:} type of rel. or action from source to middle/other entity
\item {\bfseries rat23\+:} type of rel. or action from middle/other entity to target
\item {\bfseries t\+:} timepoint
\item trailing {\bfseries \+\_\+\+:} pointer
\item trailing {\bfseries \+\_\+it\+:} pointer used as iterator
\item {\bfseries x2y\+:} map mapping xs to ys (typically a map, unordered\+\_\+map, or vector)
\end{DoxyItemize}

\subsection*{Other naming conventions }


\begin{DoxyItemize}
\item void functions are named by imperatives (e.\+g., \char`\"{}do\+\_\+this\char`\"{})
\item non-\/void functions are named by nouns (e.\+g., \char`\"{}most\+\_\+important\+\_\+thing\char`\"{}) 
\end{DoxyItemize}